\clearpage
\def\delequal{\mathrel{\ensurestackMath{\stackon[1pt]{=}{\scriptstyle\Delta}}}}
%//==============================--@--==============================//%
\subsection*{\underline{2.2} Circuito RLC-Série}
%//==============================--A--==============================//%
\subsubsection*{(a) Estabeleça a equação para a corrente ${i}$ em valores instantâneos para ${t \ge 0}$ em função do coeficiente de amortecimento ${\beta}$ e da frequência angular das oscilações não amortecidas ${\omega_0}$.
Calcule $\omega_0$.}
\label{subsubsec_a2}
\paragraph{Resposta:}
Escolhendo o caminho fechado conforme o enunciado (interruptor comutado), com o sentido coincidente com o da corrente convencionado, obtém-se, através da Lei de Indução:
$$
    \overrightarrow{\nabla} \times \overrightarrow{E} = -\dfrac{\partial}{\partial t}\overrightarrow{B} \implies i(t)\cdot (R' + R_G) + \frac{1}{C} \int i(t)\, dt + L \frac{di(t)}{dt} = 0
$$

Para obter a ilustre EDO linear de 2\textsuperscript{\underline{a}} ordem homogénea, é simplesmente necessário derivar em ordem a $t$, i.e.:
$$
    i(t)\cdot (R' + R_G) + \frac{1}{C} \int i(t)\, dt + L \frac{di(t)}{dt} = 0 \implies  L \frac{d^2i(t)}{dt^2} + \frac{di(t)}{dt}\cdot R + \frac{1}{C}\cdot i(t) = 0
$$

$$
    \iff \frac{d^2i(t)}{dt^2} + \frac{R}{L} \cdot \frac{di(t)}{dt} + \frac{1}{CL}\cdot i(t) = 0
$$

Por comparação direta com a forma canónica $\dfrac{d^2i(t)}{dt^2} + 2\beta \cdot \dfrac{di(t)}{dt} + \omega_{0}^{2}\cdot i(t) = 0$, pode-se concluir celeremente:

$$
    \begin{cases}
        \beta = \dfrac{R}{2L} \\
        \omega_0 = \dfrac{1}{\sqrt{LC}} = \dfrac{\sqrt{LC}}{LC} = 50\ \text{krad s}^{-1}
    \end{cases}
$$
\hfill \ensuremath{\Box}
%\footnotetext[2]{Do enunciado temos que $R = R' + R_G$.}
%//==============================--B--==============================//%
\subsubsection*{(b) Estabeleça as condições iniciais para o regime que se obtém para ${t \ge 0}$. Caracterize o regime forçado para ${t \ge 0}$.}
\label{subsubsec_b2}
\paragraph{Resposta:}
Caracteristicamente de sistemas reais, é de esperar que não ocorram saltos energéticos infinitos derivados de descontinuidades de $u_C(t)$ e $i_L(t)$.
$$
    p_C = \frac{d}{dt}W_e(t) \to \infty;\ \ \ p_L = \frac{d}{dt}W_m(t) \to \infty
$$

Esta situação é, naturalmente, fisicamente absurda, e não se pode evitar a conclusão óbvia de que $u_C(t)$ e $i_L(t)$ se verificam inalterados no instante de comutação do interruptor. Isto é:
$$
    u_C(0^+) = u_C(0^-);\ \ \ i_L(0^+) = i_L(0^-) 
$$

Através da Lei de Indução, salientando que se trata de um regime estacionário para $t < 0$, e que se trata de um circuito em série, concluem-se deste modo as condições iniciais:
$$
    \begin{cases}
        \bar{U}_G = \bar{U}_C + \bar{U}_L \\
        i(t) = i_L(t)
    \end{cases}
    \implies
    \begin{cases}
        u_C(0) = U_G(0) = 4\ \text{V} \\
        i(0^+) = i(0^-) = i(0) = 0\ \text{A}
    \end{cases}
$$

Assim, ao comutar o interruptor, para $t > 0$, estamos perante um regime livre da influência do gerador. 

Como esta influência foi removida, a solução dos problemas de valor inicial deve tender para zero há medida que o tempo se alastra (existem elementos dissipativos no circuito). Pelo que, naturalmente se verifica:
$$
    \begin{cases}
        %i_f(t) = \lim\limits_{t \to \infty} i(t) = 0\ \text{A}\\
        %u_f(t) = \lim\limits_{t \to \infty} u_C(t) = 0\ \text{V}
        i_f(t) = 0\ \text{A}\\
        u_f(t) = 0\ \text{V}
    \end{cases}
$$
\hfill \ensuremath{\Box}
%//==============================--C--==============================//%
\subsubsection*{(c) Discuta os tipos de solução que pode obter para o regime livre com ${R}$ variável.}
\label{subsubsec_c2}
\paragraph{Resposta:} 
Desprovido da influência do gerador, o circuito apresenta somente um regime livre que já foi introduzido, i.e.:
$$
    \dfrac{d^2i(t)}{dt^2} + 2\beta \cdot \dfrac{di(t)}{dt} + \omega_{0}^{2}\cdot i(t) = 0
$$

Esta EDO linear de 2\textsuperscript{\underline{a}} ordem homogénea é de resolução imediata recorrendo ao seu polinómio caracteristico:
$$
    s^2 + 2\beta s + \omega_0^2 = 0 \implies s = -\beta \pm \sqrt{\beta^2 - \omega_0^2}
$$

Aproveitando as relações\footnotemark[2] deduzidas na alínea \hyperref[subsubsec_a2]{\underline{2.2} (a)}, imediatamente se conclúi, ao analizar o binómio discriminante, que estamos perante três possíveis cenários:

\begin{itemize}
  \item $\beta > \omega_0 \iff R > 2\sqrt{\dfrac{L}{C}}$: estamos perante um regime aperiódico fortemente amortecido e, como tal, as raízes são reais e diferentes (a solução para $i$ seriam duas exponenciais decadentes).   
  \item $\beta = \omega_0 \iff R = 2\sqrt{\dfrac{L}{C}}$: estamos perante um regime aperiódico \textit{criticamente} amortecido (limite), neste caso temos uma raiz dupla real (a solução toma a forma de duas exponenciais decadentes em que uma é multiplicada por $t$).
  \item $\beta < \omega_0 \iff R < 2\sqrt{\dfrac{L}{C}}$: estamos perante um regime oscilatório amortecido, visto que são encontradas duas raizes complexas conjugadas (a solução é da forma de uma sinusoide multiplicada por uma exponencial decadente).
\end{itemize}
\hfill \ensuremath{\Box}

\footnotetext[2]{Especificamente $\beta = R/2L$, dado que estamos a realizar um \textit{gedankenexperiment} em que $R$ é uma variável.}
%//==============================--D--==============================//%
\subsubsection*{(d) Para ${R = 100\ \Omega}$, calcule o coeficiente de amortecimento ${\beta}$ e verifique que a solução é do tipo oscilatório amortecido. Calcule ${\omega = 2\pi/T}$ sendo ${T}$ o período de isocronismo. Verifique que: \\ $$ A_1/A_2 = A_2/A_3 = \cdots = (A_1/A_n)^{\frac{1}{n-1}} = e^{\lambda}$$ Determine $\lambda$. Determine $i(t)$ e $u_C(t)$ tendo em conta as condições iniciais estabelecidas em b).}
\label{subsubsec_d2}
\paragraph{Resposta:}
Utilizando diretamente as expressões obtidas anteriormente:
$$
    \beta = \frac{R}{2L} = \frac{5}{2\cdot 10\cdot 10^{-3}} = 5\ \text{kNp}\ s^{-1}
$$
\hfill \ensuremath{\Box}

Desta forma, encontramo-nos de facto num regime oscilatório amortecido. Tal afirmação verifica-se imediatamente com o discutido na alínea precedente juntamente o resultado obtido para a frequência natural na alínea \hyperref[subsubsec_a2]{\underline{2.2} (a)}, i.e.:
$$
    \beta < \omega_0 \iff 5\cdot 10^3 < 50\cdot 10^3 \implies [(5 < 50) \land 1] = 1 
$$
\hfill \ensuremath{\Box}

Posto isto, e tendo em conta que o regime transitório é puramente livre, temos que $i(t)$ é da forma:
\begin{equation}
    \label{eq2}
    i(t) = \mathbb{R}e\{\bar{I} e^{st} \} = Ie^{-\beta t}\cos{(\omega t + \theta)}
\end{equation}

$$
    \begin{cases}
        \omega = \sqrt{\omega_0^2 - \beta^2} \approx 49.749\ \text{krad s}^{-1} \\
        s = -\beta + j\omega = \omega_0\ e^{j\delta};\ \ \ \delta = \pi - \arctan{\left(\frac{\omega}{\beta}\right)} = 1.573\ \text{rad}
    \end{cases}
$$

Em que $\omega$ é a frequência angular das oscilações amortecidas e $\bar{I} = Ie^{j\theta}$. Assim, o período de isocronismo é facilmente deduzido:
$$
    \omega = \frac{2\pi}{T} \implies T = \frac{2\pi}{\omega} \approx 126.298\ \mu\text{s}
$$
\hfill \ensuremath{\Box}

De modo a verificar a igualdade enunciada: $A_1/A_2 = \cdots = (A_1/A_n)^{1/(n-1)} = e^{\lambda}$, com $\lambda = \beta T/2$; considera-se o facto de que de $T/2$ em $T/2$, a função regida pela sinusoide encontra ora dois zeros consecutivos, ou dois extremos consecutivos. 

Assim, denominando por $t_n$ o instante em que se verifica um extremo, temos:
$$
    t_{n+1} = t_n + \frac{T}{2} \implies
    \begin{cases}
        i(t_n) = Ie^{-\beta t_n}\cos{(\omega t_n + \theta)}\\
        i(t_{n+1}) = Ie^{-\beta t_n}e^{-\beta \frac{T}{2}}\cos{(\omega t_n + \pi + \theta)}
    \end{cases}
$$

\clearpage
Sendo então o racio entre duas amplitudes consecutivas:
$$
    \left\vert \frac{i(t_{n})}{i(t_{n+1})} \right\vert = \frac{A_{n}}{A_{n+1}} = \left\vert \frac{Ie^{-\beta t_n}\cos{(\omega t_n + \theta)}}{-Ie^{-\beta t_n}e^{-\beta \frac{T}{2}}\cos{(\omega t_n + \theta)}} \right\vert
$$

$$
    \therefore \frac{A_{n+1}}{A_{n}} = e^{\beta \frac{T}{2}} = e^{\lambda}
$$

Note-se que $t_{n+1} = t_n + T/2$ é uma progressão aritmética de razão $T/2$. Assim, temos:
$$
    t_{n} = t_1 + (n-1)\cdot \frac{T}{2} \implies t_1 = t_{n} + (1-n)\frac{T}{2}
$$

Então:
$$
    \left\vert \frac{i(t_{1})}{i(t_{n})} \right\vert = \frac{A_{1}}{A_{n}} = \left\vert \frac{Ie^{-\beta t_{n}}e^{(n-1)\beta\frac{T}{2}}\cos{(\omega t_{n} - n\pi + \pi + \theta)}}{-Ie^{-\beta t_{n}}\cos{(\omega t_{n+1} + \theta)}} \right\vert
$$

$$
    \therefore \frac{A_{1}}{A_{n}} = e^{(n-1)\beta\frac{T}{2}} \implies \left(\frac{A_{1}}{A_{n}}\right)^{1/(n-1)} = e^{\beta\frac{T}{2}} = e^{\lambda} 
$$

Finalmente, conclúi-se:
$$
    \therefore \frac{A_n}{A_{n+1}} = \left(\frac{A_1}{A_n}\right)^{1/(n-1)} = e^{\lambda};\ \ \ \lambda = \beta\frac{T}{2} \approx 315.745\ \text{mNp}
$$
\hfill \ensuremath{\Box}

Relembrando a \hyperref[eq2]{equação (3)} e as condições iniciais obtidas na alínea \hyperref[subsubsec_b2]{\underline{2.2} (b)}, encontra-se sem grande resistência o argumento $\theta$ da corrente:
$$
    i(t) = Ie^{-\beta t}\cos{(\omega t + \theta)} \implies i(0) = I\cos{(\theta)} = 0 \implies \theta = \frac{\pi}{2} + k\pi\text{, }\forall k \in \mathbb{Z} 
$$

Encontramo-nos perante um circuito em série, logo:
$$
    u_C(t) = \frac{1}{C}\int i(t)\, dt = \frac{1}{C}\int \mathbb{R}e\{\bar{I} e^{st} \}\, dt = \frac{1}{C}\mathbb{R}e\{\frac{1}{s}\bar{I} e^{st}\}
$$

$$
    \implies u_C(t) = \frac{I}{\omega_0 C}e^{-\beta t}\cos{(\omega t + \theta - \delta)}
$$

Verifica-se que no instante de comutação do interruptor $u_C(0) = 4\ \text{V}$, e assim facilmente se obtem o valor da amplitude I. Seja $k = 0$, então:
$$
    u_C(0) = \frac{I}{\omega_0 C}\cos{(\theta - \delta)} \implies I = \frac{4 \cdot \omega_0 C}{\cos{(\theta - \delta)}} \approx  8.0\ \text{mA}
$$

Em suma, temos as seguintes soluções:
$$
    i(t) = Ie^{-\beta t}\cos{(\omega t + \theta)};\ \ \ u_C(t) = \frac{I}{\omega_0 C}e^{-\beta t}\cos{(\omega t + \theta - \delta)}
$$
\hfill \ensuremath{\Box}

%//==============================--E--==============================//%
\clearpage
\subsubsection*{(e) Calcule $R_0 = R$ de modo que a solução do regime livre seja do tipo aperiódico limite. Determine $i(t)$. Determine igualmente o valor mínimo de $i\ (i_{min})$ e o instante em que ocorre $(t_{min})$.}
\label{subsubsec_e2}
\paragraph{Resposta:}
Com os resultados da alínea \hyperref[subsubsec_c2]{\underline{2.2} (c)} é de imediato calcular a resistência $R_0$ que nos leva a uma solução do polinómio característico da equação diferencial com raiz dupla (isto é, um regime aperiódico limite; dado que $\beta = \omega_0$):
$$
    R_0 = 2\sqrt{\frac{L}{C}} = 1000\ \Omega \implies \beta = 50\ \text{kNp s}^{-1}
$$

Assim, naturalmente, a solução encontrada para $i(t)$ é nos moldes:
\begin{equation}
    i(t) = (I' + t I'')e^{-\beta t}    
\end{equation}

Em seguinda, calculam-se as amplitudes $I'$ e $I''$, com uso das condições iniciais discutidas anteriormente:
$$
    i(0) = 0 \implies I' = 0\ \text{A} \implies i(t) = t I'' e^{-\beta t} 
$$

Do mesmo modo, sabe-se que $u_C(0) = 4\ \text{V}$, e por aplicação direta da Lei de Indução, temos:
$$
    u_C(0) = 4 \implies -R_0 i(0) - L \frac{di(0)}{dt} = 4 \iff \frac{di(0)}{dt} = -400\ \text{A}
$$

$$
    \frac{di(t)}{dt} = -I''(\beta t - 1)e^{-\beta t} \implies \frac{di(0)}{dt} = -400 \iff I'' = -400\ \text{A}
$$

De justa forma, obtem-se a expressão final para $i(t)$ no regime aperiódico limite:
$$
    \therefore i(t) = -400t e^{-\beta t}\text{, com }\beta = 50\ \text{kNp s}^{-1}
$$
\hfill \ensuremath{\Box}

Logicamente, visto que $i(t)$ é uma função \textit{estritamente} decrescente até ao seu valor mínimo e que após passar este ponto tende para zero (não obstante dado o regime livre), obtemos trivialmente o seu valor mínimo através do zero da sua 1\textsuperscript{\underline{a}} derivada.

Como nos lembra a Análise real:
$$
    \frac{di(t)}{dt} = -I''(\beta t - 1)e^{-\beta t} = 0 \iff t = \frac{1}{\beta} = \frac{2L}{R_0} = 20\ \mu\text{s}
$$

Então, por fim:
$$
    \begin{cases}
        t_{\text{min}} = 20\ \mu\text{s}\\
        i_{\text{min}} = i(t_{\text{min}}) = \dfrac{-1}{125e} \approx -2.943\ \text{mA}
    \end{cases}
$$
\hfill \ensuremath{\Box}
%//==============================--@--==============================//%